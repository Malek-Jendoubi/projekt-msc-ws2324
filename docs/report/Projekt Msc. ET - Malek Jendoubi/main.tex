% Statt "de" kann hier auch "en" stehen, wenn die Arbeit auf Englisch verfasst wird
% Entweder "darkstyle" oder "lightstyle"
\documentclass[en, darkstyle]{unirostock}
% Für die Quellenangaben, weitere Informationen https://de.overleaf.com/learn/latex/Bibliography_management_with_biblatex
\usepackage[backend=biber,style=alphabetic,maxbibnames=99,backref=true,citestyle=alphabetic]{biblatex}
\addbibresource{lit.bib}

% Welche Bedingungen gibt es für die Verbreitung der Arbeit?
% creative-commons: Creative Commons Lizenz, Namensnennung - Weitergabe erlaubt unter gleichen Bedingungen 
% private: Veröffentlichung und Veränderung nur nach Rücksprache mit dem Autor
\license{creative-commons}

\author{Malek Jendoubi}
\enrolmentnumber{221202139} % Matrikelnummer

\title{Feasibility Study of Human Activity Recognition with Air Pressure Sensors}
\type{Project Msc. Elektrotechnik}

\course{Elektrotechnik}
\supervisor{Dr.-Ing. Florian Grützmacher}

\faculty{Fakultät für Informatik und Elektrotechnik}
\institute{Institut für Elektrotechnik}
\workinggroup{Lehrstuhl für Eingebettete Systeme}

\begin{document}
\maketitle

\pagenumbering{Roman}

\tableofcontents % Inhaltsverzeichnis.
\clearpage

\setstretch{1.263}
\pagenumbering{gobble}
\clearpage

\pagenumbering{arabic} % Ab hier folgt die "arabische" Seitennummerierung.

% Im Ordner chapter sollte für jedes Kapitel eine Datei angelegt und hier eingebunden werden.
% Das erhöht die Übersicht.
\chapter{Introduction}
\printmyminitoc{1} % Wenn es mehr als einen Abschnitt gibt

This chapter offers a short introduction into the subject, including the motivations, the state of the art, and the goals of the study.

\section{Motivation}
In recent technological advancements, air pressure sensors have emerged as a promising modality for human activity detection when strategically affixed to the human body. This master’s project focuses on harnessing the potential of air pressure sensors, offering a novel perspective on activity monitoring. 

Similar to accelerometers and magnetometers, air pressure sensors fabricated in Micro-Electro-Mechanical Systems (MEMS) technology feature a low power consumption, making them particularly suitable for wearable applications where energy efficiency is paramount. 

Additionally, the availability of new sensor technology with enhanced accuracy opens avenues for more precise and reliable human activity recognition. 

Notably, the use of air pressure sensors in human activity detection represents a relatively unexplored territory in research, providing an opportunity for the student to contribute to the growing field of sensor-based human-computer interaction.

\section{State of the Art}
Air pressure sensors have previously been used in detecting Human Movement, especially when moving between rooms.\cite{patel2008detecting}

Here talk about the state of the art.

\subsection{Human Activity Recognition}
Here talk about human activity recognition. 

\subsection{Novelty of Air Pressure Sensors in Activity Detection}


\section{Overview}

In recent technological advancements, air pressure sensors have emerged as a promising method for human activity detection when strategically affixed to the human body. 

Similar to accelerometers and magnetometers, air pressure sensors fabricated in Micro-Electro-Mechanical Systems (MEMS) technology feature a low power consumption, making them particularly suitable for wearable applications where energy efficiency is paramount. 

Additionally, the availability of new sensor technology with enhanced accuracy opens avenues for more precise and reliable human activity recognition. 

\section{Goal of the study}

This master’s project focuses on figuring out the potential of air pressure sensors. A microcontroller board should be programmed to read sensor data via I2C or SPI from an attached state-of-the-art MEMS air pressure sensor. We have chosen the Bosch Sensortec BMP581 pressure sensor.

The acquired data should be transmitted via the microcontrollers integrated Bluetooth Low Energy (BLE) chip to a recording device. 

In order to verify the functionality of the setup, first data sets of human activities like walking on level ground, up stairs, and downstairs with a few different placements of the sensors on the human body, e.g., foot and wrist, should be recorded. 

Finally, the quality of acquired data sets and the different sensor placements should be discussed regarding their usefulness for the anticipated activity recognition purpose.
\chapter{Proposed Solution}
\printmyminitoc{1} % Wenn es mehr als einen Abschnitt gibt
\section{Proposed Solution}
Our solution consists of an nRF52832 microcontroller reading the values from a BMP581 Pressure Sensor through its I2C interface. The Microcontroller then sends the pressure sensor values as well as a timestamp of the values via Bluetooth Low Energy to client subscribed to a GATT characteristic.

\begin{figure}[ht]
    \centering
    \includegraphics[width=0.4\linewidth]{images/BMP581_fritzing.png}
    \caption{Proposed solution}
    \label{fig:solution}
\end{figure}

We can then either read the values with nRF Connect BLE App, store them in a log file, and then feed them to a python script that will parse the pressure values.

\section{Server Side}
\par
Here talk about the Server Side.

\subsection{Hardware}
\par
Here talk about the hardware part. 

\subsubsection{Pressure sensor: BMP581}
\par
Here talk about the Pressure sensor: BMP581. 

\subsubsection{Microcontroller: nRF52}
\par
Here talk about Microcontroller: nRF52. 

\subsubsection{Additions to the experimental setup}
\par
Here talk about the additions to the experimental setup to help in taking the measurements. 


\section{Client Side}
\par
Here talk about the Client Side.

\subsubsection{Hardware}
\par
Here talk about the PC BLE interface. 

\subsubsection{Software}
\par
Description of the Python client.

\subsection{Experimental Setup}
\par
Here is a description of the experimental setup on MS Visio.

\chapter{Results}
\printmyminitoc{1} % Wenn es mehr als einen Abschnitt gibt
Here we present our experimental setup, as well as the results that we were able to visualize, we will then interpret these results as to their usefulness in recognizing human activity.

\section{Experiment}
Here we will describe the experimental setup as well as the experiment protocol and the parameters we will control for.

\subsection{Experiments Protocol}
Description of the different results

\section{Results}


\chapter{Conclusion}
% \printmyminitoc{1} % Wenn es mehr als einen Abschnitt gibt

\section{Evaluation of the results}


\section{Issues and possible improvements}



\clearpage
\listoffigures % Abbildungsverzeichnis
\printbibliography % Quellenverzeichnis
\clearpage
\pagenumbering{gobble}
\chapter*{Erklärung}
Hiermit erkläre ich, dass ich die vorliegende Masterarbeit selbständig verfasst und keine anderen als die angegebenen Quellen und Hilfsmittel benutzt habe.

Alle Stellen, die wörtlich oder sinngemäß aus Veröffentlichungen entnommen sind, sind als solche kenntlich gemacht.

Die Arbeit ist noch nicht veröffentlicht und ist in ähnlicher oder gleicher Weise noch nicht als Prüfungsleistung zur Anerkennung oder Bewertung vorgelegt worden.

Rostock, den 30.02.2024
\\
\\
\\

\begin{Form}
  \digsigfield{7cm}{3cm}{Unterschrift}
\end{Form}
\end{document}