\chapter{Introduction}
\printmyminitoc{1} % Wenn es mehr als einen Abschnitt gibt

This chapter offers a short introduction into the subject, including the motivations, the state of the art, and the goals of the study.

\section{Motivation}
In recent technological advancements, air pressure sensors have emerged as a promising modality for human activity detection when strategically affixed to the human body. This master’s project focuses on harnessing the potential of air pressure sensors, offering a novel perspective on activity monitoring. 

Similar to accelerometers and magnetometers, air pressure sensors fabricated in Micro-Electro-Mechanical Systems (MEMS) technology feature a low power consumption, making them particularly suitable for wearable applications where energy efficiency is paramount. 

Additionally, the availability of new sensor technology with enhanced accuracy opens avenues for more precise and reliable human activity recognition. 

Notably, the use of air pressure sensors in human activity detection represents a relatively unexplored territory in research, providing an opportunity for the student to contribute to the growing field of sensor-based human-computer interaction.

\section{State of the Art}
Air pressure sensors have previously been used in detecting Human Movement, especially when moving between rooms.\cite{patel2008detecting}

Here talk about the state of the art.

\subsection{Human Activity Recognition}
Here talk about human activity recognition. 

\subsection{Novelty of Air Pressure Sensors in Activity Detection}


\section{Overview}

In recent technological advancements, air pressure sensors have emerged as a promising method for human activity detection when strategically affixed to the human body. 

Similar to accelerometers and magnetometers, air pressure sensors fabricated in Micro-Electro-Mechanical Systems (MEMS) technology feature a low power consumption, making them particularly suitable for wearable applications where energy efficiency is paramount. 

Additionally, the availability of new sensor technology with enhanced accuracy opens avenues for more precise and reliable human activity recognition. 

\section{Goal of the study}

This master’s project focuses on figuring out the potential of air pressure sensors. A microcontroller board should be programmed to read sensor data via I2C or SPI from an attached state-of-the-art MEMS air pressure sensor. We have chosen the Bosch Sensortec BMP581 pressure sensor.

The acquired data should be transmitted via the microcontrollers integrated Bluetooth Low Energy (BLE) chip to a recording device. 

In order to verify the functionality of the setup, first data sets of human activities like walking on level ground, up stairs, and downstairs with a few different placements of the sensors on the human body, e.g., foot and wrist, should be recorded. 

Finally, the quality of acquired data sets and the different sensor placements should be discussed regarding their usefulness for the anticipated activity recognition purpose.